\documentclass{book}
\usepackage[utf8]{inputenc}
\usepackage{xepersian}
\settextfont{XB Niloofar}
\begin{document}
\chapter{هازل}
در طول سومین حمله، هازل تقریبا یک تخته سنگ را خورد. او به مه می نگریست، و از خود می پرسید  پرواز در امتداد محدوده یک کوه مضحک چقدر سخت است که زنگ های هشدار به صدا در آمدند.

نیکو از دکل جلویی کشتی پرنده فریاد زد:«توقف سخته.»

لئو در پشت سکان، حرکتی ناگهانی به چرخ داد. آرگو دو به طرف چپ چرخید، و پاروهای هواییش مانند ردیفی از چاقوها در ابرها فرورفتند.

هازل مرتکب اشتباه نگریستن از روی نرده شد. جسمی تاریک و کروی به او برخورد کرد. او اندیشید چرا ماه به طرف ما می یاد؟ سپس جیغ کشید و روی عرشه افتاد. سنگ بزرگ آن قدر نزدیک از بالای سرش رد شد که موهایش را روی صورتش ریخت.

ترق.

دکل جلویی فرو ریخت و بادبان، تیرک‌ها و نیکو روی عرشه سقوط کردند. تخته سنگ، به بزرگی وانت بار، انگار که جایی دیگر کار مهمی دارد درون مه افتاد.

«نیکو.» هازل، درحالیکه لئو کشتی را پایین می برد، چهاردست و پا به سمت او رفت.

نیکو زیر لب گفت: «حالم خوبه.» و با لگدی بادبان را از پایش دور کرد.

هازل به او کمک کرد بایستد، و آن دو به طرف قسمت جلوی کشی تلو تلو خوردند.

ابرها زمانی کافی برای اینکه نوک کوه زیر پایشان آشکار شود از هم جدا شدند. نوک تیز سنگ سیاه از دامنه سبز خزه‌ای بالا آمده بود. یک خدای کوه بر قله ایستاده بود، جاسون به آن‌ها می‌گفت نومینا مونتانوم. یا به زبان یونانی، اورائ. آن‌ها هرچه صدا زده می‌شدند، زشت بودند.

این یکی، مانند آن‌هایی که با آن روبرو شده بودند، روی پوستی به سیاهی و تیرگی سنگ بازالت، یک پیراهن سفید ساده پوشیده بود. او حدود بیست فوت قد داشت و به شدت عضلانی بود و با ریشی آزاد به رنگ سفید، موهایی خشن، و نگاهی وحشی در چشمانش مانند زاهدی دیوانه به نظر می رسید. او چیزی فریاد زد که هازل نفهمید، اما به وضوح خوش آمد گویی نبود. او با دستان خالیش، تکه سنگی دیگر را از کوهش بلند کرد و شروع به در آوردن آن به شکل یک توپ کرد.

منظره محو شد، اما هنگامی که خدای کوهی دوباره فریاد زد، بقیه نومینا ها از فاصله‌ای دور پاسخ دادند، صداهای آن‌ها د سراسر دره طنین انداخت.

لئو از پشت سکان فریاد زد: «خدیان احمق کوهی. این سومین باریه که من باید دکل رو عوض کنم. فکر میکنین دکل ها روی درخت ها رشد میکنن؟»

نیکو اخم کرد. «دکل ها از درخت‌ها درست می‌شن.»

«نکته این نیست.» لئو یکی از اهرم‌هایش را گرفت و آن را در یک دایره چرخاند. چند فوت آن طرف‌تر، دریچه‌ای در عرشه باز شد. یک استوانه برنز آسمانی بالا آمد. هازل قبل از اینکه استوانه در هوا تخلیه شد و دوازده گوی فلزی با دنباله آتش سبز را بیرون بیاندازد فقط وقت داشت تا گوش‌هایش را بگیرد. در وسط هوا از گوی‌ها میخ‌هایی رویید، مانند پره‌های بالگرد، و آن‌ها در مه سقوط کردند. لحظه‌ای بعد، مجموعه‌ای از صداهای ترق ترق، که با غرش‌های خشمگین خدایان کوهی دنبال می‌شد، در امتداد کوه منفجر شد.

لئو فریاد زد: «هاه.»

هازل طبق دو رویایی قبلیشان، حدس زد که جدیدترین سلاح لئو، نومینا را تنها آزار داده است. تخته سنگ دیگری به سمت پهلوی راست کشتیشان در هوا صفیر کشید.

نیکو فریاد زد: «ما رو از اینجا بیرون ببر.»

لئو حرف‌های زشتی در مورد نومینا زد، اما فرمان را چرخاند. موتورها به وزوز افتادند. دکل جادویی خودش را محکم کرد و کشتی به طور مارپیچ به طرف چپ رفت. آرگو دو سرعت گرفت، و همانطور که آن‌ها در دو روز قبل این کار را کرده بودند، به شمال غربی پناه برد.

هازل آرام نگرفت تا اینکه آن‌ها از کوه خارج شدند. مه زدوده شد. زیر پای آن‌ها، نور خورشید، ییلاقات ایتالیایی، تپه‌های ناهموار سبز و مزارع زرینی که با کالیفرنیای شمالی چندان متفاوت نبودند را روشن کرده بود. هازل می‌توانست تصور کند که به طرف خانه‌اش در کمپ ژوپیتر در هوا پرواز می‌کند.

این فکر روی سینه‌اش سنگینی می‌کرد. کمپ ژوپیتر تنها به مدت نه ماه خانه او بود، تا اینکه نیکو او را از جهان زیرزمین بازگردانده بود. اما او بیشتر از زادگاهش در نیواورلئان، و به طور گستاخانه‌ای آلاسکا، جایی که او در سال هزار و نهصد و چهل و دو در آن مرده بود، دلتنگ آنجا نشده بود.

او دلتنگ تخت خوابش در سربازخانه گروهان پنجم شده بود. او دلتنگ شام‌های آن تالار کثیف، با ارواح بادی که سینی‌ها را به هوا می‌انداختند و سربازهایی که در مورد بازی‌های جنگی شوخی میکردند شده بود. او می‌خواست درحالیکه دستانش را با فرانک ژانگ گرفته بود، در خیابان‌های رم جدید سرگردان باشد. او میخواست یک بار دختر عادی بودن را تجربه کند، با یک دوست پسر شیرین و دلسوز.

بیشتر از همه او می‌خواست احساس کند ایمن است. او از همیشه هراسان و نگران بودن خسته شده بود.
او درحالیکه نیکو تراشه‌های دکل را از بازویش در می‌آورد و لئو به دکمه‌‌های روی میز فرمان کشتی ضربه می‌زد روی عرشه عقبی ایستاد.

لئو گفت: «خب، تخیلی افتضاحه. باید بقیه رو بیدار کنم؟»

هازل وسوسه شد که بگوید بله، اما خدمه دیگر کشتی شیفت شب را برداشته بودند و مستحق استراحتشان بودند. آن‌ها از دفاع از کشتی خسته شده بودند. به نظر می‌رسید که هر چند ساعت یک هیولای رومی تصمیم می‌گیرد که آٰگو دو مانند خوراکی خوشمزه است.

چند هفته قبل، هازل باور نمی‌کرد که کسی بتواند در میان حمله نومینا بخوابد، اما اکنون او تصور میکرد که دوستانش در طبقات پایینی خر و پف می‌کنند. او هربار که فرصتی برای خوابیدن پیدا میکرد، مانند بیماری در کما می‌خوابید.

هازل گفت: «اونا نیاز به استراحت دان. ما باید خودمون یه راه دیگه پیدا کنیم.»

«هاه.»  لئو به سمت نمایشگرش اخم کرد. او در لباس کار تکه تکه‌اش و شلوار جین روغن آلودش، به نظر می‌رسید که یک مسابقه کشتی با لوکوموتیو را باخته است. لئو از وقتی که دوستانش پرسی و آنابث در تارتاروس سقوط کردند، بی وقفه کار میکرد. او عصبانی‌تر حرکت می‌کرد و حتی بیشتر از معمول تحریک میشد.

هازل در مورد او نگران بود. اما بخشی از وجودش با تغییر برجسته‌تر شده بود. هر بار که لئو لبخند می‌زد و شوخی می‌کرد، خیلی شبیه پدربزرگش،‌سامی به نظر می‌رسید، اولین دوست پسر هازل، در سال هزار و نهصد و چهل و دو.

آه چرا زندگی او باید اینقدر پیچیده می‌بود؟

لئو غرغر کرد: «یه راه دیگه. تو راهی می‌بینی؟»

نقشه‌ای از ایتالیا روی نمایشگر لئو می‌درخشید. کوهستان آپاناین از وسط کشور چکمه مانند می گذشت. یک نقطه سبز برای آٰرگو دو روی سطح غربی دامنه چشمک می‌زد. چند صد پا در شمال روم. مسیر آن‌ها باید ساده می‌بود. آن‌ها باید به جایی که در زمان یونانی به آن اپیروس می‌گفتند می‌رسیدند و معبدی قدیمی که خانه هادس ( یا پلوتو، آن طور که رومی ها او را صدا می‌کردند. یا آن طور که هازل دوست داشت در موردش بیاندیشد:‌بدترین پدر غایب دنیا.) نامیده می‌شد را می‌یافتند.

برای رسیدن به اپیروس،‌آن‌ها باید مستقیم به شرق می‌رفتند، روی کوهستان آپاناین و در امتداد دریای آدریانتیک، اما نمیتوانستند این کار را بکنند. هربار که آن ها می کوشیدند از کمر ایتالیا رد شوند، خدایان کوهی حمله می‌کردند.



\end{document}
